\listfiles

\documentclass[11.5pt,a4paper]{article}

\usepackage{amsthm,amssymb,amsmath}
\usepackage{wasysym}
\usepackage{mathtools}

\newcommand\persiangloss[2]{#1\dotfill\lr{#2}\\}

\newcommand{\nocontentsline}[3]{}
\newcommand{\tocless}[2]{\bgroup\let\addcontentsline=\nocontentsline#1{#2}\egroup}
\usepackage[bottom]{footmisc}
\usepackage{indentfirst}

\usepackage{caption}

\usepackage{graphicx}
\usepackage{subcaption}
\usepackage{array}
\usepackage{adjustbox}
\usepackage{tablefootnote}
\usepackage{amsfonts}
\usepackage{amssymb}
\usepackage{yfonts}

\usepackage[scr=euler,bb=ams]{mathalfa}

\usepackage{xcolor,colortbl}
\definecolor{Gray}{gray}{0.90}
\definecolor{LGray}{gray}{0.95}

\usepackage[pagebackref=false,colorlinks,linkcolor=blue,citecolor=magenta]{hyperref}

\usepackage[a4paper]{geometry} 
\geometry{a4paper,tmargin=3.5cm, bmargin=2.5cm, lmargin=2cm, rmargin=2.5cm, headheight=3em, headsep=1.5cm, footskip=1cm} 

\usepackage{xepersian}
\settextfont[Scale=1]{B Nazanin}
%\setlatintextfont[Scale=1]{Times New Roman}

%\settextfont[Scale=1.1]{B Zar}

%\DefaultMathsDigits
\setdigitfont{XB Zar}

\defpersianfont\titr[Scale=1]{B Titr}
%\defpersianfont\nastaliq[Scale=1.5]{IranNastaliq}
%\defpersianfont\traffic[Scale=1]{B Traffic}
%\defpersianfont\yekan[Scale=1]{B Yekan}
%\defpersianfont\traffic[Scale=1]{XB Roya}
%\defpersianfont\yekan[Scale=1]{XB Kayhan}
%%%%%%%%%%%%%%%%%%%%%%%%%%%%%%%%%%%%%%%%%%%%%%%%%%%
\usepackage{zref-perpage}
\zmakeperpage{footnote}



\begin{document}

\thispagestyle{empty}
\vspace*{-28mm}
\centerline{\includegraphics[height=5cm]{Imgs/ceit_logo.png}}

\begin{center}
%دستوری برای کم کردن فاصله بین لوگو و خط پایین آن
\vspace{-2mm}
{\LARGE
{
دانشکده مهندسی کامپیوتر و فن‌آوری اطلاعات\\	
دانشگاه صنعتی امیرکبیر	
}
%دستوری برای تعیین فاصله بین دو خط
\\[2.1cm]
}

{\large
\textbf{گزارش تمرین دوم درس مدل‌های احتمالاتی گرافی}
\\[2cm]

استاد درس:
\\[.5cm]
{\Large
دکتر نیک‌آبادی}
\\[1.5cm]
\large 
نام دانشجو:
\\[.5cm]
{\Large
احمد اسدی}
\\[.5cm]
۹۴۱۳۱۰۹۱
\\[1.5cm]
}
%دستوری برای تعیین فاصله بین خطوط (نه دو خط) و تا وقتی که مقدار آن تغییر نکند، فاصله بین خطوط، همین مقدار است.

{\large
فروردین ۱۳۹۵
}
\end{center}

\newpage
\baselineskip=1cm
\tocless\tableofcontents

\newpage
\baselineskip=0.75cm
\pagenumbering{arabic}

%%%%%%%%%%%%%%%%%%%%%%%%%%%%%%%%%%%%%%%%%%%%%%%%%%%%%%%%%%%%%%%%%%%%%%%%%%%%%%%%%%%
\section{}

%%%%%%%%%%%%%%%%%%%%%%%%%%%%%%%%%%%%%%%%%%%%%%%%%%%%%%%%%%%%%%%%%%%%%%%%%%%%%%%%%%%
%%%%%%%%%%%%%%%%%%%%%%%%%%%%%%%%%%%%%%%%%%%%%%%%%%%%%%%%%%%%%%%%%%%%%%%%%%%%%%%%%%%
%%%%%%%%%%%%%%%%%%%%%%%%%%%%%%%%%%%%%%%%%%%%%%%%%%%%%%%%%%%%%%%%%%%%%%%%%%%%%%%%%%%
%%%%%%%%%%%%%%%%%%%%%%%%%%%%%%%%%%%%%%%%%%%%%%%%%%%%%%%%%%%%%%%%%%%%%%%%%%%%%%%%%%%
%%%%%%%%%%%%%%%%%%%%%%%%%%%%%%%%%%%%%%%%%%%%%%%%%%%%%%%%%%%%%%%%%%%%%%%%%%%%%%%%%%%
%%%%%%%%%%%%%%%%%%%%%%%%%%%%%%%%%%%%%%%%%%%%%%%%%%%%%%%%%%%%%%%%%%%%%%%%%%%%%%%%%%%
%%%%%%%%%%%%%%%%%%%%%%%%%%%%%%%%%%%%%%%%%%%%%%%%%%%%%%%%%%%%%%%%%%%%%%%%%%%%%%%%%%%
%%%%%%%%%%%%%%%%%%%%%%%%%%%%%%%%%%%%%%%%%%%%%%%%%%%%%%%%%%%%%%%%%%%%%%%%%%%%%%%%%%%
%%%%%%%%%%%%%%%%%%%%%%%%%%%%%%%%%%%%%%%%%%%%%%%%%%%%%%%%%%%%%%%%%%%%%%%%%%%%%%%%%%%
%%%%%%%%%%%%%%%%%%%%%%%%%%%%%%%%%%%%%%%%%%%%%%%%%%%%%%%%%%%%%%%%%%%%%%%%%%%%%%%%%%%
\vfill
\section{توضیحات}
\begin{itemize}
\item [*]به دلیل زمان اجرای طولانی الگوریتم و همین‌طور مشابهت حالت دوکلاسه و بیست‌کلاسه در آزمایش و به منظور صرفه‌جویی در زمان اجرای الگوریتم، در برخی موارد، به جای گزارش نتایج حالت بیست‌کلاسه به ارائه نتایج حالت دوکلاسه بسنده کرده‌ایم. بدیهی‌ است  در این موارد، نتایج حاصله و تحلیل‌های انجام شده، همگی قابل تعمیم به حالت بیست‌کلاسه هستند.
\item [*] سورس کد مربوط به جعبه‌ابزار مورد استفاده در ضمیمه پروژه قرار داده شده است. همین‌طور می‌توانید برای دریافت اطلاعات بیشتر در مورد این جعبه‌ابزار به 
\href{http://www.openpr.org.cn/index.php/NLP-Toolkit-for-Natural-Language-Processing/43-Naive-Bayes-Classfier/View-details.html}{این آدرس }
مراجعه نمایید.

\item [*] سورس کد مربوط به پروژه در ضمیمه این گزارش ارسال شده است. همین‌طور این کد از
\href{https://github.com/ahmad-asadi/PGM/tree/master/BayesianNetwork}
{این لینک}
، قابل 
دریافت می‌باشد.
%\item [*] آدرس لینک برای دریافت کد:
%\LTR{
%\url{https://github.com/ahmad-asadi/PGM/tree/master/BayesianNetwork}
%}
\end{itemize}




\end{document} 
